\chapter{Introducción}

\section{Resumen Ejecutivo}


Este documento presenta el desarrollo de un sistema de gestión integral para talleres mecánicos, con el objetivo de optimizar procesos administrativos, mejorar la organización y facilitar la toma de decisiones mediante herramientas digitales avanzadas.

\subsection{Propósito del Proyecto}
En la actualidad, la digitalización de los procesos administrativos es un reto para muchas pequeñas y medianas empresas. La falta de un sistema eficiente puede generar desorganización y pérdida de información, afectando la operatividad. Este proyecto busca ofrecer una solución web accesible, escalable y segura que centralice la gestión de inventarios, órdenes de trabajo, facturación y clientes.

\subsection{Objetivos Clave}
Los principales objetivos del proyecto incluyen:
\begin{itemize}
	\item Implementar un sistema web que automatice tareas administrativas.
	\item Mejorar la eficiencia en la gestión de inventarios y facturación.
	\item Facilitar el análisis de datos mediante herramientas de reporte.
	\item Asegurar la accesibilidad y seguridad de la información almacenada.
\end{itemize}

\subsection{Alcance del Proyecto}
El desarrollo contempla:
\begin{itemize}
	\item Creación de una plataforma web accesible desde cualquier dispositivo.
	\item Implementación de funcionalidades clave: control de inventario, órdenes de trabajo y facturación.
	\item Integración con bases de datos seguras y estructuradas.
	\item Diseño de una interfaz intuitiva y adaptable.
\end{itemize}

\subsection{Beneficios Esperados}
La implementación del sistema proporcionará:
\begin{itemize}
	\item **Eficiencia operativa:** Reducción de tiempo en tareas administrativas.
	\item **Seguridad de datos:** Almacenamiento estructurado y encriptado.
	\item **Optimización de recursos:** Mejor control financiero y operativo.
	\item **Facilidad de acceso:** Plataforma disponible en cualquier navegador.
\end{itemize}

\section{Contexto y Justificación}

En este apartado se describe el contexto en el que se desarrolla el proyecto y se justifica su importancia. Se debe explicar la problemática que se pretende resolver, así como la relevancia del proyecto dentro de su ámbito de aplicación.

\begin{itemize}
	\item ¿Cuál es el problema o necesidad que motiva este proyecto?
	\item ¿Por qué es importante abordarlo?
	\item ¿Existen antecedentes o proyectos similares?
	\item ¿Qué impacto tiene la solución propuesta en la industria, la comunidad o el entorno en general?
\end{itemize}

Ejemplo de contenido:

\begin{quote}
	En la actualidad, la digitalización de los procesos administrativos es un reto para muchas pequeñas y medianas empresas. La falta de un sistema de gestión eficiente puede llevar a la pérdida de información, desorganización y problemas en la toma de decisiones. Este proyecto busca desarrollar un sistema integral que permita mejorar la eficiencia operativa a través de una solución basada en tecnologías web modernas.
\end{quote}

\section{Objetivos}

\subsection{Objetivo General}
El objetivo general del proyecto define el propósito central del mismo, es decir, qué se pretende lograr en términos amplios.

Ejemplo:
\begin{quote}
	Desarrollar un sistema de gestión integral para talleres mecánicos que permita el control de inventarios, órdenes de trabajo, facturación y gestión de clientes, optimizando los procesos administrativos y operativos.
\end{quote}

\subsection{Objetivos Específicos}
Los objetivos específicos detallan las acciones concretas necesarias para alcanzar el objetivo general.

Ejemplo:
\begin{itemize}
	\item Diseñar una base de datos estructurada que almacene la información de clientes, servicios y productos.
	\item Implementar una interfaz web intuitiva para la gestión de órdenes de trabajo.
	\item Desarrollar módulos de facturación y control financiero automatizados.
	\item Integrar herramientas de análisis de datos para mejorar la toma de decisiones.
\end{itemize}

\section{Alcance del Proyecto}

El alcance define los límites del proyecto, estableciendo qué aspectos serán cubiertos y cuáles quedan fuera del desarrollo.

\textbf{Incluido en el proyecto:}
\begin{itemize}
	\item Desarrollo de un sistema web accesible desde cualquier navegador.
	\item Funcionalidades de control de inventario, facturación y administración de órdenes de trabajo.
	\item Base de datos segura y estructurada para la gestión de la información.
	\item Interfaz de usuario amigable y adaptable a dispositivos móviles.
\end{itemize}

\textbf{Excluido del proyecto:}
\begin{itemize}
	\item Desarrollo de una aplicación móvil nativa.
	\item Integración con sistemas externos de terceros no especificados inicialmente.
	\item Soporte técnico post-implementación sin contrato de mantenimiento.
\end{itemize}

Este documento servirá como base para la planificación y desarrollo del proyecto, asegurando que todos los aspectos clave sean considerados desde el inicio.
