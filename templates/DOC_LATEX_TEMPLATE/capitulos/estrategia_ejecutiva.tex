\chapter{Estrategia Ejecutiva y Toma de Decisiones}

\section{Plan de Negocio y Viabilidad}

El plan de negocio establece la estrategia y modelo de operación del proyecto, asegurando su viabilidad económica y técnica.

\subsection{Modelo de Negocio}
Este proyecto se basa en un modelo de negocio sostenible con las siguientes características:
\begin{itemize}
	\item \textbf{Propuesta de Valor}: Solución eficiente y automatizada para la gestión de talleres mecánicos.
	\item \textbf{Segmento de Clientes}: Empresas de mantenimiento, mecánicos independientes y concesionarios de automóviles.
	\item \textbf{Canales de Distribución}: Plataforma en la nube accesible desde web y dispositivos móviles.
	\item \textbf{Fuente de Ingresos}: Venta de licencias, suscripciones mensuales y personalización del software.
\end{itemize}

\subsection{Análisis de Viabilidad}
Para evaluar la viabilidad del proyecto, se realiza un estudio de costos y retorno de inversión (\textit{ROI}).

\begin{table}[h]
	\centering
	\begin{tabular}{|l|c|}
		\hline
		\textbf{Concepto} & \textbf{Costo Estimado (USD)} \\ \hline
		Desarrollo & 10,000 \\ \hline
		Infraestructura & 5,000 \\ \hline
		Marketing & 2,000 \\ \hline
		Operación Inicial & 3,000 \\ \hline
		\textbf{Total} & \textbf{20,000} \\ \hline
	\end{tabular}
	\caption{Costos iniciales del proyecto}
\end{table}

\textbf{Retorno de Inversión}: Se espera recuperar la inversión en un plazo de 12 meses con un crecimiento del 15\% mensual en suscripciones.



\section{Riesgos y Mitigaciones}

Todo proyecto conlleva riesgos que deben ser identificados y gestionados para minimizar su impacto.

\subsection{Análisis de Riesgos}
Los principales riesgos del proyecto se presentan en la siguiente tabla:

\begin{table}[h]
	\centering
	\begin{tabular}{|l|c|c|}
		\hline
		\textbf{Riesgo} & \textbf{Impacto} & \textbf{Mitigación} \\ \hline
		Baja adopción de usuarios & Alto & Campañas de marketing dirigidas \\ \hline
		Problemas de seguridad & Crítico & Auditorías periódicas y cifrado de datos \\ \hline
		Retrasos en el desarrollo & Medio & Metodología ágil para entregas incrementales \\ \hline
		Falta de financiamiento & Alto & Alternativas como crowdfunding o inversores \\ \hline
	\end{tabular}
	\caption{Matriz de Riesgos y Mitigaciones}
\end{table}

\subsection{Estrategia de Mitigación}
Las estrategias para minimizar riesgos incluyen:
\begin{itemize}
	\item **Plan de contingencia tecnológica:** Implementar copias de seguridad y redundancia.
	\item **Análisis continuo de mercado:** Adaptarse a nuevas tendencias.
	\item **Evaluación financiera trimestral:** Ajuste de costos y optimización de recursos.
\end{itemize}



\section{Indicadores Clave de Desempeño (KPIs)}

Los \textit{Key Performance Indicators} (KPIs) permiten evaluar el rendimiento del proyecto y medir su éxito.

\subsection{Métricas Financieras}
\begin{itemize}
	\item \textbf{Ingresos recurrentes mensuales (MRR)}: Total de ingresos por suscripciones mensuales.
	\item \textbf{Costo de adquisición de clientes (CAC)}: Gasto en marketing dividido entre el número de nuevos clientes.
	\item \textbf{Retorno de inversión (ROI)}: Relación entre ingresos y costos.
\end{itemize}

\subsection{Métricas de Usuario}
\begin{itemize}
	\item \textbf{Tasa de retención}: Porcentaje de clientes que siguen usando el servicio después de 6 meses.
	\item \textbf{Tiempo de uso promedio}: Tiempo que los usuarios pasan en la plataforma.
	\item \textbf{Net Promoter Score (NPS)}: Satisfacción de los clientes basada en encuestas.
\end{itemize}

\subsection{Métricas Operativas}
\begin{itemize}
	\item \textbf{Tiempos de respuesta del sistema}: Latencia promedio en milisegundos.
	\item \textbf{Uptime}: Disponibilidad del sistema en porcentaje (\%).
	\item \textbf{Número de incidencias reportadas}: Cantidad de problemas técnicos identificados y resueltos.
\end{itemize}


\section{Impacto y Sostenibilidad}

El proyecto no solo tiene un impacto tecnológico, sino también **económico, social y ambiental**. Se destacan los siguientes puntos:

\subsection{Impacto Económico}
\begin{itemize}
	\item Reducción de costos operativos al automatizar procesos manuales.
	\item Creación de oportunidades para nuevos empleos en desarrollo y soporte.
\end{itemize}

\subsection{Impacto Social}
\begin{itemize}
	\item Facilita el acceso a herramientas digitales a pequeñas y medianas empresas.
	\item Mejora la calidad del servicio al cliente en la industria de talleres mecánicos.
\end{itemize}

\subsection{Sostenibilidad Ambiental}
\begin{itemize}
	\item Reducción del uso de papel mediante digitalización.
	\item Implementación de servidores eficientes con bajo consumo energético.
\end{itemize}
