\chapter{Estrategia Administrativa}

\section{Roles y Responsabilidades}

En esta sección se definen los roles clave dentro del proyecto y las responsabilidades de cada uno. La estructura organizativa es crucial para garantizar una ejecución eficiente del proyecto.

\subsection{Estructura del Equipo}
El equipo está compuesto por los siguientes roles:

\begin{table}[h]
	\centering
	\begin{tabular}{|c|l|}
		\hline
		\textbf{Rol} & \textbf{Responsabilidades} \\ \hline
		Gerente de Proyecto & Supervisión general, toma de decisiones estratégicas \\ \hline
		Desarrollador & Implementación del software y mantenimiento del código \\ \hline
		Diseñador UX/UI & Creación de interfaces gráficas amigables \\ \hline
		Tester / QA & Pruebas y control de calidad \\ \hline
		Administrador de Infraestructura & Gestión de servidores y redes \\ \hline
	\end{tabular}
	\caption{Roles y responsabilidades del equipo}
\end{table}

\subsection{Matriz RACI}
La matriz RACI ayuda a clarificar la asignación de tareas dentro del equipo:

\begin{table}[h]
	\centering
	\begin{tabular}{|l|c|c|c|c|}
		\hline
		\textbf{Tarea} & \textbf{Gerente} & \textbf{Desarrollador} & \textbf{Diseñador} & \textbf{QA} \\ \hline
		Planificación & R & A & & \\ \hline
		Desarrollo Backend & & R & & \\ \hline
		Diseño UI & & & R & \\ \hline
		Pruebas & & & & R \\ \hline
	\end{tabular}
	\caption{Matriz RACI de Responsabilidades}
\end{table}



\section{Gestión de Recursos Humanos}

El éxito del proyecto depende de la correcta gestión del equipo, asegurando una asignación eficiente de tareas y la capacitación adecuada.

\subsection{Proceso de Selección}
El proceso de selección de personal sigue estos pasos:
\begin{enumerate}
	\item Definición de requisitos del puesto
	\item Publicación de vacantes
	\item Evaluación de candidatos
	\item Entrevistas técnicas y personales
	\item Contratación y capacitación
\end{enumerate}

\subsection{Plan de Capacitación}
Para garantizar la competencia técnica del equipo, se implementará el siguiente plan de capacitación:

\begin{table}[h]
	\centering
	\begin{tabular}{|c|l|c|}
		\hline
		\textbf{Área} & \textbf{Temas} & \textbf{Duración} \\ \hline
		Desarrollo & Python, Flask, PostgreSQL & 3 meses \\ \hline
		Infraestructura & Kubernetes, Docker, Podman & 2 meses \\ \hline
		Seguridad & Ciberseguridad, prácticas de DevSecOps & 1 mes \\ \hline
	\end{tabular}
	\caption{Plan de Capacitación}
\end{table}



\section{Presupuesto y Financiamiento}

Aquí se detallan los costos del proyecto y las fuentes de financiamiento.

\subsection{Desglose de Costos}
Se ha realizado una estimación de costos para garantizar la viabilidad del proyecto.

\begin{table}[h]
	\centering
	\begin{tabular}{|l|c|}
		\hline
		\textbf{Concepto} & \textbf{Costo Estimado (USD)} \\ \hline
		Desarrollo de Software & 10,000 \\ \hline
		Infraestructura & 5,000 \\ \hline
		Licencias de Software & 2,000 \\ \hline
		Capacitación & 3,000 \\ \hline
		Otros Gastos & 1,500 \\ \hline
		\textbf{Total} & \textbf{21,500} \\ \hline
	\end{tabular}
	\caption{Presupuesto estimado del proyecto}
\end{table}

\subsection{Fuentes de Financiamiento}
El proyecto será financiado a través de:

\begin{itemize}
	\item Inversión de capital propio
	\item Subvenciones gubernamentales
	\item Acuerdos con inversores
	\item Crowdfunding
\end{itemize}
