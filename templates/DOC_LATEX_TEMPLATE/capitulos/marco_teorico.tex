\chapter{Marco Teórico y Referencias}

\section{Antecedentes y Fundamentos}

En este apartado se presentan los antecedentes del proyecto y los fundamentos teóricos en los que se basa. Se debe incluir información relevante de investigaciones previas, proyectos similares o tecnologías utilizadas.

\subsection{Antecedentes}

Los antecedentes permiten contextualizar el proyecto dentro de un marco histórico y técnico. Se deben responder preguntas como:

\begin{itemize}
	\item ¿Existen proyectos similares? ¿Cómo han abordado el problema?
	\item ¿Qué tecnologías se han utilizado previamente?
	\item ¿Cuáles han sido los principales desafíos en proyectos similares?
\end{itemize}

Ejemplo:
\begin{quote}
	En los últimos años, múltiples empresas han desarrollado sistemas de gestión empresarial basados en la nube para optimizar sus procesos administrativos. Sin embargo, muchos de estos sistemas están diseñados para grandes empresas y no se adaptan bien a negocios más pequeños. Este proyecto busca cerrar esa brecha mediante el desarrollo de una solución escalable y accesible.
\end{quote}



\subsection{Fundamentos Teóricos}

Aquí se deben describir los conceptos y principios fundamentales que sustentan el desarrollo del proyecto. Se recomienda abordar:

\begin{itemize}
	\item Modelos, teorías o metodologías en los que se basa el proyecto.
	\item Explicación de conceptos clave, como tecnologías o frameworks utilizados.
	\item Normativas y estándares relacionados con el proyecto.
\end{itemize}

Ejemplo:
\begin{quote}
	La arquitectura MVC (Modelo-Vista-Controlador) es un patrón de diseño ampliamente utilizado en el desarrollo de software. Permite separar la lógica del negocio, la interfaz de usuario y el control de flujo, facilitando la escalabilidad y el mantenimiento del sistema.
\end{quote}

\subsection{Comparación con Soluciones Existentes}

Para evaluar la viabilidad del proyecto, se compararon sus características con otras soluciones disponibles en el mercado:

\begin{table}[h]
	\centering
	\begin{tabular}{|l|c|c|c|}
		\hline
		\textbf{Característica} & \textbf{Proyecto} & \textbf{Alternativa 1} & \textbf{Alternativa 2} \\ \hline
		Arquitectura Modular & \ding{51} & \ding{55} & \ding{51} \\ \hline
		Integración con API & \ding{51} & \ding{51} & \ding{55} \\ \hline
		Costo & Bajo & Medio & Alto \\ \hline
		Código Abierto & \ding{51} & \ding{55} & \ding{55} \\ \hline
		Facilidad de Uso & Alta & Media & Baja \\ \hline
	\end{tabular}
	\caption{Comparación con soluciones existentes}
\end{table}

Esta tabla demuestra que el proyecto tiene ventajas competitivas en términos de arquitectura modular, integración con APIs y accesibilidad en términos de costo y facilidad de uso.

\section{Gobernanza y Normativas}

El proyecto sigue una serie de normativas y mejores prácticas para garantizar seguridad, escalabilidad y cumplimiento regulatorio.

\begin{itemize}
	\item **ISO 27001:** Implementación de estándares de seguridad de la información.
	\item **GDPR:** Cumplimiento con normativas de protección de datos personales.
	\item **PMBOK:** Gestión de proyectos basada en las mejores prácticas de PMI.
	\item **Metodología DevSecOps:** Enfoque de desarrollo seguro e integración continua.
\end{itemize}




\section{Bibliografía Utilizada}

Este apartado presenta las referencias bibliográficas utilizadas en el desarrollo del proyecto. Para gestionar las citas correctamente, se recomienda el uso de un archivo de bibliografía en formato BibTeX.

Ejemplo de referencia en el documento:


Según \cite{pressman2005ingenieria}, el proceso de desarrollo de software debe basarse en un enfoque estructurado y bien documentado.

La teoría de computabilidad fue introducida por \citep{turing1936}.