\chapter{Resultados y Evaluación}

\section{Análisis de Resultados}

Tras la implementación del proyecto, se realizó un análisis de los resultados obtenidos en función de los objetivos planteados. Los principales indicadores evaluados fueron el desempeño del sistema, la aceptación por parte de los usuarios y la estabilidad operativa.

\subsection{Comparación con los Objetivos Iniciales}
La siguiente tabla muestra una comparación entre los objetivos propuestos y los resultados obtenidos:

\begin{table}[h]
	\centering
	\begin{tabular}{|p{5cm}|p{5cm}|c|}
		\hline
		\textbf{Objetivo} & \textbf{Resultado} & \textbf{Cumplimiento} \\ \hline
		Implementar un sistema accesible en la nube & Plataforma web funcional y accesible & Cumplido \\ \hline
		Optimizar tiempos de respuesta & Reducción del 30\% en consultas a la base de datos & Cumplido \\ \hline
		Mejorar la experiencia del usuario & Encuestas con satisfacción del 85\% & Parcialmente cumplido \\ \hline
		Integrar módulos de análisis & Implementación de panel de métricas & Cumplido \\ \hline
	\end{tabular}
	\caption{Comparación entre Objetivos y Resultados}
\end{table}


\subsection{Desempeño del Sistema}
Se realizaron pruebas de carga y rendimiento, obteniendo los siguientes resultados:

\begin{itemize}
	\item **Tiempo de respuesta promedio:** 120ms por solicitud.
	\item **Capacidad de concurrencia:** Hasta 1,000 usuarios simultáneos sin degradación del rendimiento.
	\item **Disponibilidad:** 99.8\% en el primer mes de operación.
\end{itemize}



\section{Problemas Encontrados y Soluciones}

Durante el desarrollo e implementación del sistema, se identificaron diversos problemas. A continuación, se presentan los principales inconvenientes y las soluciones aplicadas:

\begin{table}[h]
	\centering
	\begin{tabular}{|p{5cm}|p{5cm}|}
		\hline
		\textbf{Problema} & \textbf{Solución} \\ \hline
		Latencia alta en consultas a la base de datos & Implementación de Redis como caché \\ \hline
		Dificultades en la integración con API externa & Creación de middleware para manejo de errores \\ \hline
		Reportes lentos con grandes volúmenes de datos & Optimización de consultas SQL y uso de índices \\ \hline
		Rechazo inicial de usuarios por cambios en la interfaz & Capacitación y documentación accesible \\ \hline
	\end{tabular}
	\caption{Problemas y Soluciones}
\end{table}

\subsection{Lecciones Aprendidas}
Las principales lecciones obtenidas durante el proyecto incluyen:

\begin{itemize}
	\item La optimización temprana del backend evita problemas de rendimiento en producción.
	\item La capacitación a usuarios es clave para una adopción más rápida del sistema.
	\item La modularidad en el código facilita la escalabilidad y mantenimiento.
\end{itemize}



\section{Medición del Éxito}

Para evaluar el éxito del proyecto, se utilizaron diversos indicadores de desempeño (\textit{KPIs}) en áreas clave:

\subsection{Satisfacción del Usuario}
Se realizaron encuestas de satisfacción entre los usuarios, obteniendo:

\begin{itemize}
	\item **Nivel de satisfacción general:** 85\%
	\item **Facilidad de uso:** 4.3/5
	\item **Recomendación del producto:** 88\% de usuarios lo recomendarían.
\end{itemize}

\subsection{Crecimiento y Adopción}
Se midió el crecimiento del sistema en términos de usuarios y engagement:

\begin{table}[h]
	\centering
	\begin{tabular}{|l|c|}
		\hline
		\textbf{Indicador} & \textbf{Valor} \\ \hline
		Usuarios registrados en el primer mes & 1,200 \\ \hline
		Uso promedio diario del sistema & 3.5 horas por usuario \\ \hline
		Incremento mensual de usuarios & 20\% \\ \hline
	\end{tabular}
	\caption{Crecimiento y Uso del Sistema}
\end{table}

\subsection{Desempeño Financiero}
En términos financieros, el proyecto logró recuperar su inversión en 9 meses, superando la estimación inicial de 12 meses.



