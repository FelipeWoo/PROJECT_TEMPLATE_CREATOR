\chapter{Implementación Técnica}

\section{Diseño y Arquitectura}

El diseño y la arquitectura del sistema definen cómo se estructuran los componentes y cómo interactúan entre sí. Se ha adoptado una arquitectura modular para mejorar la escalabilidad y el mantenimiento del proyecto.

\subsection{Arquitectura del Sistema}
El sistema se basa en una arquitectura cliente-servidor con separación de responsabilidades:

\begin{itemize}
	\item \textbf{Frontend:} Interfaz de usuario desarrollada con React.
	\item \textbf{Backend:} API desarrollada en FastAPI que gestiona la lógica del negocio.
	\item \textbf{Base de datos:} PostgreSQL para almacenamiento estructurado y Redis para almacenamiento en caché.
	\item \textbf{Infraestructura:} Contenedores con Podman y orquestación con Kubernetes.
\end{itemize}

\subsection{Diagrama de Arquitectura}
El siguiente diagrama muestra la arquitectura general del sistema:

\begin{center}
	\begin{otherlanguage}{english} % Desactiva babel temporalmente
		\begin{tikzpicture}[
			node distance=2cm and 1.5cm,
			every node/.style={draw, rounded corners=3pt, minimum width=4.5cm, minimum height=1cm, align=center, font=\small},
			every path/.style={draw, thick, -latex}
			]
			% Nodos principales
			\node (client) {Cliente (React)};
			\node[right=of client] (api) {API (FastAPI)};
			\node[right=of api] (db) {Base de Datos (PostgreSQL)};
			\node[below=of api] (cache) {Cache (Redis)};
			
			% Flechas de conexión
			\draw[->] (client.east) -- (api.west);
			\draw[->] (api.east) -- (db.west);
			\draw[->] (api.south) -- (cache.north);
		\end{tikzpicture}
	\end{otherlanguage} % Reactivar babel en español
\end{center}



\section{Tecnologías Utilizadas}

Para garantizar eficiencia y escalabilidad, se han seleccionado las siguientes tecnologías:

\subsection{Lenguajes de Programación}
\begin{itemize}
	\item \textbf{Python:} Para el desarrollo del backend con FastAPI.
	\item \textbf{JavaScript:} Para el desarrollo del frontend con React.
\end{itemize}

\subsection{Bases de Datos y Almacenamiento}
\begin{itemize}
	\item \textbf{PostgreSQL:} Base de datos relacional para almacenar datos estructurados.
	\item \textbf{Redis:} Caché en memoria para mejorar el rendimiento.
\end{itemize}

\subsection{Herramientas de Desarrollo}
\begin{itemize}
	\item \textbf{Podman y Kubernetes:} Para la gestión de contenedores y orquestación.
	\item \textbf{GitHub y GitHub Actions:} Para control de versiones e integración continua.
	\item \textbf{Obsidian y LaTeX:} Para documentación estructurada del proyecto.
\end{itemize}



\section{Desarrollo y Ejecución}

Esta sección describe el proceso de desarrollo, desde la configuración inicial hasta la ejecución del sistema.

\subsection{Flujo de Desarrollo}
El desarrollo sigue una metodología ágil con entregas iterativas. El proceso incluye:

\begin{enumerate}
	\item Definición de requisitos y planificación de sprints.
	\item Desarrollo de módulos individuales (backend, frontend, base de datos).
	\item Pruebas unitarias y de integración.
	\item Despliegue en entorno de pruebas y ajustes finales.
\end{enumerate}

\subsection{Configuración del Entorno}
Para configurar el entorno de desarrollo, se deben seguir los siguientes pasos:

\begin{verbatim}
	# Clonar el repositorio
	git clone https://github.com/usuario/proyecto.git
	cd proyecto
	
	# Crear entorno virtual en Python
	python -m venv venv
	source venv/bin/activate  # En Windows: venv\Scripts\activate
	
	# Instalar dependencias
	pip install -r requirements.txt
\end{verbatim}

\subsection{Despliegue en Producción}
El despliegue se realiza con Kubernetes y Podman. Se usa un `deployment.yaml` para definir los contenedores:

\begin{verbatim}
	apiVersion: apps/v1
	kind: Deployment
	metadata:
	name: fastapi-app
	spec:
	replicas: 2
	template:
	spec:
	containers:
	- name: app
	image: usuario/fastapi-app
	ports:
	- containerPort: 8000
\end{verbatim}

Una vez definido, se ejecuta:

\begin{verbatim}
	kubectl apply -f deployment.yaml
\end{verbatim}

